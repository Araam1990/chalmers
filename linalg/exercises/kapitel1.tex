\documentclass{article}
\title{Linjär Algebra}
\author{Pølse}
\begin{document}
    \maketitle
    \tableofcontents
    \pagebreak
    \section{Geometriska vektorer}
        \textbf{Avsnitt 1.1 och 1.2}
        \subsection{}
            \begin{enumerate}
                \item[a)]
                \begin{math}
                    1+1
                \end{math}
                \item[b)]
                \begin{math}
                    1+1
                \end{math}
                \item[c)]
                \begin{math}
                    1+1
                \end{math}
            \end{enumerate}
        \subsection{}
            \begin{enumerate}
                \item[a)]
                \begin{math}
                    1+1
                \end{math}
                \item[b)]
                \begin{math}
                    1+1
                \end{math}
                \item[c)]
                \begin{math}
                    1+1
                \end{math}
            \end{enumerate}
    \section{Matriser}
    \section{Geometriska linjära avbildningar}
    \section{Rummet $R^n$}
    \section{Linjära ekvationssytem}
    \section{Determinant}
    \section{Baser}
    \section{Egenvärden och vektorer}
    \section{Grafer och grannmatriser}
\end{document}

\subsection{}
\begin{math}

\end{math}